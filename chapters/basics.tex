\chapter{Basics}\label{ch:basics}

\q{We see what we are, we don't see what there is.}{Unknown Social Psychologist}

At the foundation of an effectively connecting communication lies the ability to see reality for what it is, which sounds much, much easier than it actually is.
The ability to entangle each other's stories, to not get into a mess, being confused of what's yours and what's mine.
But in order to get there, we might need the help of some kind ``walking stick''; something which is not the end goal, but helps us to get there.
This helping tool will be the mechanics, the external representation of something which supports us in enabling an internal skill; once we managed to do so, we can let go of it again, as the embodied principle is still there.

It's like when painting a house: We won't get far unless we build a scaffold\footnote{A scaffold is the metal construction made out of many poles to help construction worker climb up and access the whole facade of a house.}.
So we build that first, which takes quite some time and effort.
Once it's done, we finally get to the real work: The painting of the house.
Now, a lot of people are pretty proud of their scaffold construction, and don't dare to tear it down, but that's exactly what's necessary in order to be able to see the beautiful result of your painting.
The most extremist people would not even paint a single stroke, as they became experts in building scaffolds.

The ``unconfusion'' also starts by not seeing each other due to acting out of pain and the other person's pain also being touched; instead of pain, many people also use synonymous like ``being triggered'' or ``pushing each other's buttons''.
If I am unaware of one of my pain points being touched, I might complain and yell at you, basically offering a present of disconnection.
The mistake now many people do is to accept that present, symbolizing that they take what I've said personal, not seeing me, and instead letting their own pain (insecurity, or whatever it might be) being touched.
The goal would be to see what's really going on, that the other person is confused, expressing emotions from an old wound, and thus not accepting the gift being offered, keeping our stories separate, and ultimately staying grounded, being able to give empathy and preserving harmony in our connection.

I once visited a Yoga studio for the first time, and me being a bit anxious in new environments was not sure where to get the props for the practice.
The teacher saw my wandering around and out of nowhere suddenly yelled at me that I need to get my stuff from there, and then go over here, in the most commanding and rude voice I've heard for a long time.
At first, I was at shock, not knowing how comes, but fortunately I was quickly able to reason what's going on, that this was not personally directed to me, but that's just the way she might be.
I imagined how her life must be if this is the way her inner chat, the way she talks to herself is all the time.
A sudden wave of pain, pity, empathy and care overcame me; a slight touch of sadness and tears while I looked at her.
``I see you'', and that's why I wouldn't let her story become my story.

\section{Terminology}\label{sec:terminology}

\q{"NO is always a YES to something else.}{Marshall B. Rosenberg}

Every system comes with a set of custom words, a jargon, which is specific to it and usually not understood outside of this context.
Additionally, some already familiar words might have been assigned either a slightly or even a more severe different meaning.
In order to get started we therefor need to first clear things up and get clarity of what we mean by using one word or the other.

\begin{itemize}
    \item The \textbf{giraffe} symbolizes all the qualities we strive for in our way of communicating.
    It is the land mammal with the biggest heart, and thus listens empathetically (giraffe ears) and speaks in terms of needs, feelings and requests, instead of judgments, diagnosis and commands (giraffe language).
    Because it is so tall, it also sees above all, sees further into the future, and thinks more long-term, not engaging in strategies of manipulation and guilt-inducing which will maybe work temporarily, but the price to pay, the damage done to the relationship is way too high.
    \begin{itemize}
        \item Once we start learning, we are still little \textbf{baby giraffe}s, and as such we might have adapted a strange way of talking, sounding like a robot, rather artificial, and this leads temporarily to even more disconnection with you and the other person.
        Be patient with yourself, be compassionate, and know that you are still learning, still growing up.
        \item The \textbf{Street Giraffe} is someone who uses a more ordinary, a more regular way of talking, to be able to adapt to the currently exposed environment, as otherwise it would alienate the connection.
        Think of street gangsters: You can't use highly sophisticated language of emotions and needs, but still you are able to express them by using an ordinary language fitting for the other.
        \item The way we talk to others is basically identical the way we talk to ourselves, thus when we change the way we talk to ourselves (also called thinking), we ultimately change the way we interact with others, which we refer to as the \textbf{Inner Giraffe}, a form of self-love and self-empathy.
    \end{itemize}
    \item The second animal is the \textbf{jackal}, which in some societies is a feared creature, yet our intention is not to label him as something bad, cruel.
    There are actually no jackals, just people who have not yet learned how to connect deeper to themselves, and thus are basically ``sheep in a wolfskin''.
    It is a language full of interpretations (and confusing them with the absolute reality), judgments (what's right or wrong, good or bad), thinking in terms of strategies and how to make people what they want in a harmful way.
    Not being connected to their own needs, and stating demands while threatening with emotional punishment, guilt, shame, and other life alienating methods.
    Once the Giraffe ears have grown big enough, we will be able to hear underneath every Jackal expression the hidden Giraffe, yet this requires some time and effort to learn; enjoy the journey until then.
    \begin{itemize}
        \item We all sometimes carry our own jackal with us in form of an \textbf{inner jackal}, the internal critical voice telling us how bad we are, how stupid, how wrong, how incapable, how evil, how insufficient; you name it.
        Here as well, we can try to use empathy to listen what is actually going on underneath, because there is a beauty behind this terrible voice, which just wants to be revealed.
        Instead of seeing him as an enemy, we can try to figure something out about ourselves, about our needs, and with some gentle kindness listen carefully, sooth that little scared creature inside of us, which has not yet learned how to express itself differently.
        Once we learn to do that with ourselves, to be able to see what's going on underneath, to see reality for what it truly is, we will also be able to do it with others more easily.
        And a very important advice here: Never try to hear the inner jackals of others, only needs and emotions, as otherwise this is guarantee to homo- and suicide.
        \item In the spiritual new-age scene there is some kind of toxic positivity present which makes it impossible for them to live their lives fully, with all the colors it can express itself, going full rainbow.
        In such a (hedonistic) world, where only good is what good does, and a forceful way of constant happiness towards everything and everyone (usually out of fear of any other emotional expression, thus true love shines only through courage), having any kind of judgment is considered to be wrong.
        Feeling annoyed, irritated, and even hateful towards a person, a thing, or a subject is considered wrong, and some would not even allow these kind of thoughts arising within them, although they are so much human to have.
        Suppressing part of our human nature leads to tension, anxiety, and all the consequences ranging from mental disorders to chronic physical illness.
        Our tendency to judge our judgments can be therefore called \textbf{meta-jackals}, a way of hating hate, criticising criticism, and trying to extinguish fire by using fire\footnote{The metaphor of the stupidity of trying to extinguish fire by fire is not entirely correct, I'm well aware of that, as when trying to stop forrest fires, it is indeed a useful approach, but I consider this as rather the exception than the rule.}.
    \end{itemize}
    \item An \textbf{observation} is an objective fact everyone can see and agree to, whereas an \textbf{interpretation} is seeing reality through a subjective lens and often contains a moral evaluation of right and wrong.
    \item We also often confuse when we want to satisfy one of our basic, human \textbf{needs}, that there is only one single way to get it, via only one \textbf{strategy}.
    A need is something very fundamental, we all humans share, they are universal, and always beautiful expressions of life, like the need for respect, connection, warmth, and others.
    It is part of a bigger picture, a long-term goal, and a strategy is a concrete action to get it, something executable, an operation.
    \item Sympathy, empathy, and compassion are all emotional reactions to someone's suffering, but have some significant differences.
    \textbf{Sympathy} is the most basic emotional reaction, usually pity, without any deeper thought or personal reflection.
    Usually there is no action being taken or consideration beyond your feelings, and also feeling pity about someone might make them feel demoralized and patronized.
    \textbf{Empathy} goes a step further where you put yourself into his position, and making him feel validated and understood.
    There is no action involved, nothing to be improved, just acknowledging the other's emotions, which can also potentially lead to feeling overwhelmed by stepping into this experience, which isn't helpful if the other person is looking for a calm presence.
    Lastly, the highest form is \textbf{compassion} which has the beneficial aspects of both previous sensations: you feel the emotions of the other person, understand what he is feeling, yet keeping a healthy, emotional distance to avoid feeling overwhelmed, and additionally it involves action.
    \begin{itemize}
        \item The German language holds two more relevant differentiations, those of ``Mitleiden'' (with-suffering) and ``Mitfühlen'' (with-feeling).
        Imagine a person is drowning in a river, then the first one would be to jump in there and drown with that person, now we are both going to die, both being overwhelmed by the tragic emotion.
        The second way of relating is like offering a stick to that drowning person, keeping myself stable and safe, while acknowledging and helping the other.
    \end{itemize}
\end{itemize}

\q{With empathy, I'm fully with them, not full of them — that's sympathy.}{Marshall B. Rosenberg}

\q{Regardless of our many differences, we all have the same needs. What differs is the strategy for fulfilling these needs.}{Marshall B. Rosenberg}

\section{Holding Space}\label{sec:holding-space}

\q{Most people do not listen with the intent to understand; they listen with the intent to reply.}{Stephen Covey}

The term of holding space for someone refers to a very high art of unconditional love, acceptance and presence.
We create a safe room where emotions can be expressed and verbalized, and we become a grounded anchor for that person.
There is a very clear role assignment, and can be compared with a psychotherapy session, as this is what a good therapist also does, to hold space for their clients.

The holding aspect has to do with allowing the other person to go deep into his experience, maybe even losing himself in it, while knowing that we stay stable and are reliable, like a rock, stable and firm.
We can of course also physically hold the person, embrace, and hug him, yet not interfering, or manipulating in any direction.
We don't want the person to stop crying when he is crying, or shift to focus to anything else but what there is.
We are not a guide, we are not the expert, we are just there next to them, fully present, holding a hand and going on a journey together.

The space represents the non-judgmental attitude of emptiness which accepts and allows whatever there is.
Any hate and anger expressed by the person is not to be taken literally, but just an expression of something irrational, which we try to look with the deepest compassion and humility, to have the honor and be allowed and trusted to witness such rawness of a person.
We don't go into a discussion, wanting to change by giving advice, or let alone redirecting the attention to ourselves, we simply acknowledge any expression, possibly reflecting it back, and staying open and curious to go along with the other person.

\begin{exercise}{Holding Space}
    Person A thinks about a situation which is a strong enough trigger, but not too strong, so it will escalate; let's say a good 7 out of 10.
    Take a few minutes to talk about it, not only the facts about the situation, who was involved, but mostly about how it felt, what was touched in this situation, and also what's happening now while talking about it.

    Person B stays in a mode of attentive, and empathetical listening.
    No words need to be said, how wonderful is that: No pressure to do it wrong, just staying present, that's all.

    When there are small breaks, refrain from jumping in immediately, people sometimes need a break to sort their thoughts before continuing.
    Only after a longer break -- you will feel some sort of relaxation in your body when it is the right time -- you can thank person A for their sharing, and try to guess how that must have felt for that person, yet refrain from analysing or diagnosing, but simply name a few emotions that come to your mind which seem fitting, and do the same for needs which might have been violated in that experience.
\end{exercise}

\section{Observations and Interpretations}\label{sec:observations-and-interpretations}

It is important to see reality for what it is, which in this context basically means to be able to separate observations from interpretation; sounds easy, but trust me, we usually get it wrong most of the time.
An \textbf{observation} is an objective fact, something a camera could detect, therefore, it's not an observation to say ``he is happy'', but instead what's really going on is ``he is smiling'', or even better: ``The edges of his mouth moved up.''
On the other hand, any form of interpretation, analysis, or diagnosis, anything which would label something as good or bad, is a judgment\footnote{The word judgment actually means to judge it as either good or bad, whereas most people use the word solely for its negative evaluation as in condemnation.}, a form of \textbf{interpretation}.
Of course, our interpretations can be accurate at times.
However, the point is to be aware of the distinction of those two, and stay with only what truly is, as an interpretation usually says more about ourselves than it says about the other person.

\subsection{Reality Check}\label{subsec:reality-check}

We so often confuse our interpretations, our assumptions and guesses about the world and the thoughts and motivations of others, with the absolute, objective truth.
We are so used to uncritically believing our thoughts that we don't even challenge them even anymore, taking them for granted.
This is the best way to create hell on earth if you have a more negative, pessimistic attitude, as what you are will become your world.

To prevent those demons from continuing haunting you, it is advisable to use some tools to make them disappear, and that tool is ``asking questions'' and by doing so doing a reality check.
Whenever you feel insecure, insulted, threatened, mistreated, or any other discomfort in an interaction with another person, where you feel so certain about your assumption, just ask them whether it is true what you are thinking.
In most of the cases, you are wrong.
In some few cases, you might be even right.
If that's the case, there are always ways to get out of this terrible situation in a way which ends up in a more connected, vulnerable bond with the other person -- more about this in the following pages.
And in other cases, you might have the feeling that the other person is not telling the truth, saying that everything is fine, but you still don't believe it.
Well, when in doubt, you can continue holding onto your hell, and just tell yourself the other is lying, or go with the (naive?) assumption that the other person is indeed telling you the truth, and you have difficulties changing your mind about it.
Ultimately, you will never have certainty in what the other is really thinking, they themselves might not be aware of it.
So it is up to you to create a world which benefits you the most.

Whenever I go by car, driving on the highway, and someone cuts my line, putting me in danger, I usually would think what an asshole that is, how he (an unknown person who shows anti-social behavior like this must of course be a man) dares to take himself more important than me?!
(Yet what does that say about me, the assumption that he believes that?
Using myself as a prototype, my own motivation why I would do that, and projecting it on him.)
This kind of thinking makes me feel tiny and helpless, powerless.
The world is against me, is unfair; no one plays according to the rules, and others take advantage of me.
I put myself in a place of being the victim, solely based on my own interpretation of what happened.
Alternatively, I started to create a different world, a world where he would drive his pregnant wife to the hospital, and he is in a desperate rush for her to give birth to their firstborn child.
This story I create in my mind makes me smile, and willingly gives him the right to go first.
I feel I have helped good people out when they were in need, I feel more connected with them, with the world, with myself.
You see, I have no way to know for sure what happened, I will never ever know that's really happening, thus it is totally up to me how my mind will fill the gaps of the unknown, enabling me to construct the world by projecting my own thoughts onto these kind of ambiguous situations.
Think for yourself, in what kind of world do you want to life: One of empathy, kindness, and helping each other, or one of hatred, bitterness and loneliness?

\section{Apologizing and Forgiveness}\label{sec:apologizing-and-forgiveness}

In our culture, forgiving is almost considered a sacred act, which makes sense taking into account that it comes from a Christian background.
We would never even dare to imagine that this could be in any way harmful.
And especially the act of saying sorry, to apologize, is something we take for granted as being a virtue.
Good people say sorry when they have done something wrong, so the other person can forgive and give their blessings.

The reason why it is considered an unhealthy, harmful thing to use this kind of language, is, that it implies wrongness.
I did something wrong, thus I need to make myself small, I feel guilty, I feel shameful, I am to blame.

To forgive does not only imply wrongness on the other side, but it also creates a power hierarchy:
We are forgiving the other, we are like god, knowing what's right and wrong and give absolution because we hold that absolute wisdom.
We are not connected equally to each other anymore, and in some cases we may be even using it as a means of manipulation.

A simple solution to that is, instead of forgiving, simply don't judge in the first place.
So there is nothing that needs to be forgiven, and see the beauty underneath, how that person tragically expressed their need in a way that would cause harm.
Of course, not all of us are that enlightened yet, so the advised alternative to saying sorry and forgive is to simply mourn about what happened; to connect to the present emotions when thinking about what happened, and to connect those with the needs which weren't met.

\subsection{Shared mourning process}\label{subsec:shared-mourning-process}

Let's have a look at how our world could look like if we use a different form of communication.
One which emphasizes needs and emotions, staying present with what is alive inside us right now, and mutual empathy and the intention of staying connected.

\begin{example}{How not to say sorry and stay present in the connection}
    \exampleItemX{A}{touches B on the belly, stroking it, with the intention of connecting through physical intimacy.}
    \exampleItem{B}{When you touch my belly I feel tensed, my muscles in that area start contract in panic. I don't know exactly why, but I guess it is because of my past, where I was chubby and felt uncomfortable about my body, and I want to feel comfortable with you, while we cuddle. Can you please stop touching me there?}
    \exampleItem{A}{Thank you for letting me know, and I mourn that my touch made you feel uncomfortable, that was not my intention. I shall of course stop stroking your belly because your well-being is important to me, you are important to me.}
    \exampleItem{B}{Thank you for seeing me and doing as I request. I can imagine that you did it because you wanted to be intimate with me, and I so much like to be intimate with you too, and I hope we find a way now where we both can meet our same need in a way we both feel comfortable and satisfied with.}
    \exampleItem{A}{I feel so much joy and aliveness hearing you saying that. Let me rub your back then.}
\end{example}

In this example, I wanted to make clear that it is not necessary to say sorry for an unfortunate event to be resolved.
B started off with an external observation (you touch my belly), as well as an internal one (muscle contraction) and an emotion (panic).
Sometimes it can help the other person when we explain to them the reason, as this will take away the burden of them trying to fill in the gaps.
To possibly project their own insecurities into the uncertainty of our motivations\footnot{When my date doesn't want to kiss me, I would think of course it is because she doesn't like me, and this might then become a self-fullfiling prophecy because I will start to avoid her. Whereas in reality she was afraid of her bad breath, and she actually does like me a lot and wants nothing more than to kiss me.}.
Considering the fact that our unconsciousness is the driving force of all our motivations, and it being vastly bigger and more influential, acting in the background (the back of our minds), it is more safe to assume that all our explanations (or let's rather say ``rationalizations'', creating reasons to hide the actual, painful reasons) are mere guesses, but never certainties.
The sentence closes with a need for comfort followed by the last part of the mechanics, the request.

\q{Always trust when people when they tell you when they like or dislike something. Never trust them when they try to explain why.}{Some Psychologist}

Now pay attention to how A is not saying sorry, but instead saying ``thank you'', appreciating the other person to have set their boundaries.
We encourage, and even celebrate it, as we see in every ``no'' the ``yes'' inside.
Still, we don't run away cowardly from our own feelings, because it of course has an impact on us, a negative one, maybe even the feeling of rejection.
Don't let the celebration cover up your own pain, but instead allow both to co-exist at the same time, expressing one after the other.
Make it explicit that you understood the request by repeating it (stop touching the belly) and reassure with a need and emphasizing the importance of the relation.

After B received enough empathy, you see that something wonderful is happening: He gives empathy from within himself, guessing the other's well-intended intention.
Seeing reality for what it is: A big, unfortunate misunderstanding, like when cats and dogs try to communicate, basically trying to express the very same thing, yet unable to meet each other due to the incompatible use of (body-)language.
The conversation closes with a big celebration of life, expressing emotions being evoked, and an immediate switch to a humorous counter-suggestion as a solution attempt.

\subsection{Conclusion of mourning}\label{subsec:conclusion-of-mourning}

We simply mourn about how our behavior stimulated uncomfortable emotions in others and the unfortunate course of events, yet we don't blame either of us.
Expressing how sad we are that we did not live up to our own needs, to keep the harmony, to maintain a joyful relationship and take care of the well-being of both of us.
See the beauty of the other person expressing their own needs, instead of hearing a complaint, blaming us for doing it wrong, and us taking that judgment and making ourselves small, punishing ourselves for our wrong-doing.
Make it visible that you feel sad and that you are in pain yourself, and how much you wish to be connected again, and meet each other's needs.

The reason why it is so difficult to stop blaming, to stop trying to find a scape goat, the one who is guilty, is, that it is even more painful to just sit with the fact that there is no one to blame.
It is just so damn tragic, and there is so much pain, and it can't be redirected to someone else in the form of hate and anger, leading to destructive behavior.
See the dedicated chapter about \secref{sec:anger} for more thoughts on this topic.

Finally, it must be acknowledged that reaching this level of sophistication seems rather unlikely, yet achievable.
Please don't stop to strive for it, as at the beginning you will more often fail\footnote{Remember that failing is part of the path to success; it is a delay, not a defeat.} with it, then you will succeed.
Whenever you feel more tensed and the situation starts to get more heated up, take a deep breath, calm down your nervous system, and go slowly.
Create the space for you two be more present, more mindful and connect deeper with what's actually going on.
Have the courage to dare, to be vulnerable, to show yourself without any guarantee of success of reciprocity.
Because \ldots it is the best we can do.
