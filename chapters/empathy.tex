\chapter{Empathy}\label{ch:empathy}

I was once sitting in the subway, overfull as usual, noisy, smelly; hate it.
A father with two of his kids entered, them being crazy loud, running around wild and starting to annoy the surrounding passengers, especially me.
The worst of it was that this guy didn't care at all.
Apathetic he was staring on the floor, not giving any attention to the kids, not trying to control them in any way, totally indifferent about his responsibility of a care taker and a good citizen, sharing public transportation and bringing some peace and order to this world we have rightfully deserved.
After a while, my anger just built up more and more, like a pressure cooker, until I couldn't help it anymore but approached him and asked whether those are his kids, and whether he considers to do something about their behavior as it is disturbing the other people.
He looked up to me, with a total empty, pale face, bubbling something about that he is sorry, and that he is not very present, his kids not knowing how to deal with the situation they are going through, and neither does he, as his wife, their mother just passed away in the hospital.
A sudden rush of extreme discomfort and shame went through my body.
I stuttered some kind of apologize, went back to my place and collapsed in my seat.
If I had only known\ldots

\q{Don't just do something, be there}{Buddha}

% TODO important: always first give empathy, never ask for it. there will be most likely a window were the other will offer it.
% ... don't rush too quickly in advice, the person will let you know, ask for it.
\q{When you need empathy, you cannot give empathy.}{Marshall B. Rosenberg}
% ... you can't give what you don't have, neither love, nor empathy. if you are empty yourself, don't force yourself to give it, it won't work. retreat instead, get a timeout, "park" the situation, and ask someone else to give YOU empathy first. or give yourself empathy. => alain du buton, 20 signs of maturity: well fed, etc.
% TODO presence, most precious, only true non-renewable resource; without opinion


\q{In empathy, you don't speak at all. You speak with the eyes. You speak with your body. If you say any words at all, it's because you are not sure you are with the person. So you may say some words. But the words are not empathy. Empathy is when the other person feels the connection with what's alive in you.}{Marshall B. Rosenberg}

Empathy is a human art which is so much necessary in this world, so much asked for, and has such a great potential to do so much good, yet at the same time, it is also done so rarely, much too rarely.
With the best intention, we so often try to analyze, to diagnose, to sympathize with our own stories, that we totally forget about the other person.
Especially men have the tendency to solve problems, because this is what they can do best, and there is of course a time when solutions are needed, just often those come too early when the person is not ready yet.
People want to first be seen in their state of being, with all their experiences and sensations, with all their joy and sadness, with all their laughter and tears.
Only once we feel seen and heard enough, we might then be open to hear suggestions, to hear advice, otherwise the pain will only get bigger, the wound deeper, of not being seen.
Especially advice can hurt, as it has the underlying implication of wrongness; ``What I'm doing/saying/feeling is wrong, therefor I need to change''.

% TODO never say what others feel. ask! and connect to a need (keep it simple, dont analyze)

\q{Tout comprendre c'est tout pardonner (to understand all is to forgive all)}{Madame de Stael}

Empathy is the art of listening, of being present, of acknowledging and acceptance whatever is there at the moment.
There is absolutely nothing to do, but to pay attention, and staying open, curiously, recognizing what's happening with the other person.
Not in order to analyze or diagnose, not even internally in your own mind, but instead maybe just also keep track of the impact of the words of the other on your own body.

\begin{exercise}{Active Listening}
    Sit with your partner comfortably, get rid of all possible distractions, including sounds and phones.
    Agree who goes first, and set a timer, for example 5 minutes.
    Before you start, take yourself some time to ground, to calm down and thus be able to reach deeper into your experience.
    Person A starts to talk from the heart, whatever is alive now, anything that has meaning, without going too much into any stories.
    Person B does nothing but listening; not a single word, just sitting still and listen.
    Observe the tendency of the mind to sometimes come up with responses, and if you do so, acknowledge them, and let them go again.
    Sometimes you might want to respond with small gestures or sounds, that might be ok, sometimes though it might manipulate A's inner journey.
    If we laugh at a joke, we might encourage, reward, and thus reinforce to continue being happy, being of service to us, instead being by himself.

    Once the timer goes off, slowly take your time to round it up, no need to abruptly end in the middle of a sentence, and give a clear sign that you are done sharing, either by drawing a circle with both of your hands or saying something like ``thank you for listening'' whereas the other can respond with ``thank you for sharing''.
    Take some time to in stillness, to process what has been said, let it sink, let it digest, and giving it therefor the meaning it deserves.
    Only then, get ready to switch roles, and repeat the same again.
\end{exercise}

I would highly advise to follow the ``Hollywood principle'' with any kind of those sharings, meaning: What happens in Hollywood, stays in love.
Never address anything that has been shared within this safe, almost sacred space.
The consequence could be that the other person needs to justify or defend himself, and the next sharing will not dare to open up again because of the potential confrontation (again).
If you really, really feel the urge to comment on or ask about something to satisfy your own curiosity, then first ask for permission whether the other person is okay with that.

% TODO jackals sometimes say things about us, we feel most scared about to be

\section{Empathy Session}\label{sec:empathy-session}

Of course empathy is not only expressed by passively observing, listening and being with the other person, intellectually and emotionally.

% TODO write an example dialogue how it should go; guessing feelins and needs

listening: reflect back.; not adding anything
You can also offer to share the impact

% TODO when giving someone an empathy session: refer back to the body, what is/where is it, do you feel it?