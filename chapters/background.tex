\chapter{Background}\label{ch:background}

\q{Everything we say is either an expression of love, a thank you, or a request for love, saying please.\footnote{Every ``thank you'' we say is a celebration of life, that life is wonderful, and every ``please'' we say, is a request to make this life wonderful.}}{Marshall Rosenberg}

Our goal as human beings is naturally to connect to life, to serve life.
For that, we need to let go of thinking in absolute moral judgments like right and wrong, true and false.
Realizing that there are no bad people, not even bad behavior.
There is some kind of bidirectional nature of how we look at others and the way we look at ourselves; the way we judge others tells more about ourselves than about the other.
We get triggered by the behaviors of others which show us the parts inside ourselves we have not yet made peace with.
To shine softness, tenderness and gentleness on those we are condemning the most and look at this specific attitude to curiously investigate what drives them, and have compassion.
We are not looking for forgiveness though, although it might seem like a desired approach it contains a certain aspect of authority and superiority, as the forgiver has to forgive the ``forgivee'', an act from top down.
To go beyond those judgments and seeing each other truly of what's going on underneath; to go beyond the aspect of communication and seeing reality for what it actually is; thus it will become some sort of \textbf{spiritual} practice.

This doesn't only include how we talk to others, but also the way we talk to ourselves, also called ``thinking'', the omnipresent inner chat we engage in; thus it will become some sort of \textbf{psychotherapy}, trying to alter the state of mind.

\q{What A says about B, says more about A than about B.}{Sebastian Körber}

\section{Motivation}\label{sec:motivation}

So why would one invest to change one's language?
Things are running smoothly anyway, so what's the big deal?

Well, for the blind ones it might be difficult to see what they are missing.
We are being raised in an environment which is hostile and unhealthy, yet it is challenging to see it because we were never exposed to anything else.
We have nothing to compare it with, and as the saying goes:
``\textit{Try to tell a fish what's water.}''
We first need to \textbf{step out}, to see what we are actually surrounded with.

We use a language which leads to misunderstanding and disconnection.
We try to do it right by saying ``you did good'', or ``I am sorry''.
We have the \textbf{best intentions} in mind, and repeat what we were told to do, yet not being able to see how harmful this actually is.
We pass on this unhealthy programming to our own children, just as we learned it from our parents, and reinforce it to the people around it.

As mentioned earlier, it's not only about yet another language tool, but nothing less than a path to seeing reality for what it is.
To relate to oneself and others from a perspective of a deep understanding.
To let go of shame, guilt, fear, blame, \ldots
Life is difficult enough, even without us adding even more \textbf{suffering} by disconnecting ourselves from life itself.

It seems to me that it is our \textbf{responsibility} as human beings to do our best to find back again a level of mutual understanding in order to find ways to work together, to collaborate, to find solutions, so we can all have our needs met, without having to go into compromises\footnote{A compromise is when no one gets what they wanted. It's like going to the cinema, not being able to decide whether to watch a love story or a SciFi movie. As a consequence of this, we watch the first 45 minutes of one, quickly change rooms, and then the other half of the other movie}.
Once we can see clearly what's actually going on, we can \textbf{unconfuse ourselves} to not mix up our needs from certain strategies.
Once we are asked for a request, we will be doing so from a place to serve life, to make this life a wonderful one.
Once we can see clearly, we don't automatically and unconsciously assume that we are god, knowing the absolute truth and judging by telling how and what other people are.
Instead, we are able to express ourselves in terms of needs, emotions, and gratitude, which makes it easy for everyone to celebrate life again.
To be there for each other, to see each other, to help each other, and yourself.

\q{Those that are the hardest to love, need it the most.}{Peaceful Warrior}

It is very important to understand the goal of learning this way of communication is NOT to change others, or manipulate them, so you can get what you want.
Some people with background in NLP (Neuro Linguistic Programming) might have this immoral goal, which usually backfires on themselves, as ``everything we do to others, we do to ourselves''.
By lying, we get used to lying, and we create hell on earth because we start to mistrust others, as we project our own dishonest nature onto others, creating a hostile environment for ourselves.
Even if you claim to do it only in your professional environment, once the character is poisoned with lies, it becomes a general attitude, leaking into all areas of life.

No, the goal is to establish a relationship based on honesty and empathy, where everyone's needs are met.
Not by convincing, to get what you want at the cost of the other, but by serving life.
By both of you doing what you are doing because of an internal wish to make life wonderful for all of us, to give freely from a place of joy.

\subsection{Personal Motivation}\label{subsec:personal-motivation}

When I was about 16 years old, all students were asked in school to write topics on the blackboard we were afraid of.
You could read common words like spiders, darkness, or exams.
I would go last, a bit hesitant, and write down something vulnerable, something scary, something that would give me the most pain in my whole life so far: being \textbf{misunderstood}\footnote{Next to being misunderstood -- being called stupid because they wouldn't understand me, and me believing those judgments, just to later figuring out they lacked the cognitive capacity to follow my thought of stream, and me lacking the language skills and self-awareness of expressing myself in a way they could understand it -- the biggest pain point was simply not being understood, leaving me feeling lonely, alienated and excluded most of my life.}.

It's difficult though to be understood if I don't understand myself even.
\textit{
    What am I actually saying about this?
    What do I actually want?
    What do I actually need?
}

How can I expect others to understand me, if I don't even \textbf{understand myself}?
Especially in moments where emotions might get overwhelming.
Being blinded, being confused, being in panic and doing things I might regret afterward.

Over the years, I learned to \textbf{self-reflect}, to investigate into myself, thus knowing often very precisely what I need, how I need to be spoken and listened to.
I so much long for empathy, so much long for being seen, for being acknowledged, yet it is so hard to receive it because we are all so freaking busy with our own issues.
So I learned my next lesson:

\q{Be the change you want to see happen, instead of trying to change everyone else.}{Ernest Troutner}

Or as I phrased it once: ``\textit{Eat your own medicine!}''
Don't demand people to be the way you want them to be, but rather be a role model.
Have compassion that most people don't know themselves how to communicate properly.
By being what you so much desire others to be, you might \textbf{inspire} them to also change their way of communicating.

\q{Seek first to understand, then to be understood.}{Stephen Covey}
