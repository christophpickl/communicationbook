\chapter{Mechanics}\label{ch:mechanics}

This chapter will guide you through the basics of a system called Nonviolent Communication.
Please do not take it too literally, as it is not supposed to be used literally in real life conversation.
It merely acts as a playground, to gain some experience what it means to connect to one's (or someone else's) emotions and needs, and to be able to differentiate between facts and interpretation.

Be warned: If followed too literally, too mechanically, it will lead to the opposite of our intention, which is a deeper connection.
You will sound alien, like a robot, difficult to connect to, and ultimately separation and more conflict, leading to reaching the opposite of the goal we actually wanted to reach.
Of course, at the beginning we need practice to acquire a skill, in that case differentiate between practice and application.

This 4-step procedure of \texttt{observations -- emotions -- needs -- request} can be adapted and used in different contexts.
For example, when expressing gratitude -instead of judging someone as a ``good boy''- the first three parts can be reused.

\section{Observation}\label{sec:observation}

\q{You are the dreamer, and you are the dream.}{A Shamanistic Priest}

We humans are most of the time blind to our own interpretations, confusing them with facts.
A \textbf{fact} is something that a camera can record, or an audio recorder when it is about sound, or somehow measured and repeated by others.
Something that is absolute and objective, independent of who perceives it when and in which context.

We can observe that someone has a tear on their face, but we can't observe him to be sad; that's a diagnosis.
We can observe that our partner is looking somewhere else while we are talking, but we can't observe him ignoring us; that's an interpretation.
We can observe someone declining our offer to go out, but we can't observe his reasons; that's an analysis.
We can observe someone having hair of a certain length, but we can't observe \textit{long} hair; that's a judgment.

Our interpretation/diagnose/analysis might be accurate in some cases; yes.
Still, it is important to be very much aware of the difference to objective facts.
Doing this requires a tremendous amount of \textbf{awareness} at the moment, of what's going on in the inside of us, and express it in a more careful, mindful, and precise language.

The word \textbf{judgment} is commonly used only when declaring something as bad, as a synonym for condemnation.
But it is more correct to use it in a more general sense, as in giving something a (moral) evaluation in good and bad.
Whether something is too much, or too little, or as in the example stated above long or short.
Saying something like ``you did this well'' is therefore also a judgment, one we are very much used to use.
It implies that I'm a god-like being, knowing what's good and bad (or what's the absolute right amount, when saying someone is too much or too little of something) and putting myself above that person.

Instead, we prefer to separate the distinct phenomena of observation and interpretation, by using sentences like the following:

\begin{itemize}
    \item ``\textit{I notice that you are looking to the side, I create a story in my head that you are not interested in what I'm saying. Is that so?}'' -- It's always a good idea to do a reality-check of one's assumptions based on our own fears and insecurities, as the other person might be listening carefully just avoiding eye contact to focus better.
    \item ``\textit{I heard you saying 'not yet', and I can imagine that you are very busy at the moment.}'' -- Paraphrasing others is always a good way to get objectivity in one's observations and avoid interpreting something.
    \item ``\textit{I see your shoes and jacket on the floor, and assume you are low on energy and too lazy to keep things tidy.}'' -- The room is not dirty, as ``dirty'' can mean different things for different people, and we explicitly express our assumption which is based on our own values and standards.
\end{itemize}

Note: In social psychology there is a phenomenon called ``fundamental attribution error'', which states that we too often overestimate internal, and underestimate situational factors, when explaining people's behavior.
Because someone left their clothes on the ground, it doesn't mean that they are lazy (a disposition in their personality, set in stone, permanent).
More often, it's just because of them being exhausted, as they didn't get proper sleep of the construction site outdoors.

\section{Emotion}\label{sec:emotion}

TODOTODOTODOTODOTODOTODOTODOTODOTODOTODOTODOTODOTODOTODOTODOTODOTODOTODOTODOTODO

not "i feel that you...", that's an interpretation.
rule of thumb: when another person is needed for it, it's not an emotion
to every emotion, there is a need underneath.

% connect every emotion with a need (with the speed of life)
% emotions are many things, but in this context, for us, emotions are a compass telling us something about our needs, whether they are met or not. so therefore any emotion is great and welcome, they are all friends, especially those most uncomfortable ones.
% emotions not bad; neither thinking; appropriate (good slave, bad master); see it more nuances (good way of think/feel); not extremes (over-think/feel; yet: use best tools available

\section{Need}\label{sec:need}

\q{When we understand the needs that motivate our own and others behavior, we have no enemies.}{Marshall B. Rosenberg}

not in the "needy, selfish" sense

basically owning everything.

universal for all humans; we can connect through them
there is always a possibility that both parties can have both their needs be fulfilled (think win-win); not necessarily the specific strategies they had in mind; but first it requires us to listen openly, curiously, trying to figure out the other person's needs.

strategy: milk-breast confusion story.
my personal story: need to cuddle with her. no, not "her", but i need connectoin/intimacy (through cuddles). also possible with someone else (even in another way).

\section{Request}\label{sec:request}

Do it only to serve life, not due to guilt, shame, (fear of) punishment, wanting something back, ... it might work, but there will be a big price to pay for it.

\q{Please do as I requested, only if you can do so with the joy of a little child feeding a hungry duck. Please do not do as I request if there is any taint of fear of punishment if you don't. Please do not do as I request to buy my love, that, is hoping that I will love you more if you do. Please do not do as I request if you will feel guilty if you don't. Please do not do as I request if you will feel shameful. And certainly do not do as I request out of any sense of duty or obligation.}{Marshall B. Rosenberg}

\q{To practice the process of conflict resolution, we must completely abandon the goal of getting people to do what we want.}{Marshall B. Rosenberg}

positive, actionable
can be said "no" to (not a command/demand)

% TODO REQUESTS are not demands, a "no" is always welcome; not to get what you want (manipulating the other; that's not the purpose of this); it's about connecting what's alive, what's needed, to each other, and then find a way together so everyone's needs get met (out of empathy and celebration, not out of fear of punishment/manipulation wanting something back/guilt/shame/...). "like a young boy feeding a duck"



